% PAKETI

% geometrija stranice
% margine postavljene kao sto i pise u obrascu DR.SC.08
\usepackage[margin=25mm,centering]{geometry} 

% prihvaca tex kod kao UTF-8 omogućavajući
% pisanje znakova s dijakriticima unutar datoteke
% npr. možemo pisati 'č' umjesto č
% PAZITI DA SU DATOTEKE STVARNO SPREMLJENE KAO UTF-8
% nije korisno ako se namjerava koristiti 'latexdiff'
\usepackage[utf8]{inputenc}

% omogućiti ispisivanje naših znakova u dokumentu 
\usepackage[T1]{fontenc}

% koristi novu verziju fonta (Latin Modern)
%\usepackage{lmodern}

% Palatino font (blizu Arial-a)
%\usepackage[sc]{mathpazo}
\usepackage{uarial}

% Times New Roman
\usepackage{mathptmx}
% ili
%\usepackage{txfonts}

% koristan paket kad stvaranja pdf-a
\usepackage[activate={true,nocompatibility},
	final,
	tracking=true,
	kerning=true,
	spacing=true,
	factor=1100,
	stretch=10,
	shrink=10]{microtype}
\microtypecontext{spacing=nonfrench}
%\usepackage{microtype}

% babel !
\usepackage[croatian, english]{babel}
\usepackage{csquotes}
%\usepackage{hyphenat}
 

% captioni ce biti centrirani te će im ime biti podebljano. 
\usepackage[center,labelfont=bf]{caption}

%center command without adding spaces before and after
\newenvironment{nscenter}
 {\parskip=0pt\par\nopagebreak\centering}
 {\par\noindent\ignorespacesafterend}

% koristan paket za pisanje url-ova u tekstu
\usepackage{url}

% paketi za oblikovanje tablica
\usepackage{booktabs}
\usepackage{multirow}
\usepackage{array}
\newcolumntype{t}{>{\ttfamily}c}

% paketi za oblikovanje lista
\usepackage[ampersand]{easylist}
\usepackage{enumitem}

% boje, bojice, jeee :D
\usepackage{color}
\usepackage[table]{xcolor}
\definecolor{dark-red}{rgb}{0.4,0.15,0.15}
\definecolor{dark-blue}{rgb}{0.15,0.15,0.4}
\definecolor{medium-blue}{rgb}{0,0,0.5}
\definecolor{black}{rgb}{0,0,0}
\definecolor{fgcolor}{rgb}{0.345, 0.345, 0.345} %needed for R


% dodatni simboli
\usepackage{pifont}

% glossaries
\usepackage{supertabular}
\usepackage[nomain,acronym,nonumberlist,style=list,translate=false]{glossaries}


% prilagođavanje oblika naslova poglavalja (chapter),
% potpoglavlja (section) i odjeljka (subsection)
\usepackage{titlesec}
	\titleformat{\chapter}[display]
	{\normalfont \centering}%format
	{\fontsize{10}{12} \selectfont \uppercase{chapter} \thechapter}%label
	{0.5ex}%sep
	{\fontsize{18}{20} \selectfont \bfseries \MakeUppercase} %before-title-code
	[\vspace{-0.5ex} \rule{\textwidth}{0.3pt}] %after-title-code
  \titlespacing*{\chapter}{0pt}{0pt}{6pt}	

	\titleformat{\section}
	{\normalfont \fontsize{16}{18} \selectfont \bfseries}%format
	{\thesection}%label
	{5pt}%sep
	{}%before-title-code
	\titlespacing*{\section}{0pt}{18pt}{6pt}
	
	\titleformat{\subsection}
	{\normalfont \fontsize{14}{16} \selectfont \bfseries}%format
	{\thesubsection}%label
	{5pt}%sep
	{}%before-title-code
	\titlespacing*{\subsection}{0pt}{12pt}{6pt}

% crtanje !
\usepackage{tikz}

% verbatim env.
\usepackage{moreverb}

% multi tables/figures
\usepackage{subcaption}
\usepackage{float}
\usepackage{makecell}
\usepackage{longtable}
\usepackage{colortbl}
\usepackage{tabu}
\usepackage{threeparttable}
\usepackage{threeparttablex}
\usepackage[normalem]{ulem}
\usepackage{wrapfig}

%source code
\usepackage{listings}
\lstset{
  language=Java,
  basicstyle=\small,
  breaklines=true
  }

% uklucivanje slika
\usepackage{graphicx}
\usepackage{epstopdf}

% matematika
\usepackage{amssymb ,amsmath, amsthm, amsfonts}
\usepackage{enumitem}
\allowdisplaybreaks[1]
\usepackage{bbm}
\usepackage{xfrac} % xfrac je koristan paket za umetanje razlomaka unutar teksta

% headeri i footeri
\usepackage{fancyhdr}

% hack za fancyhdr i problem s headerheight-om
% http://nw360.blogspot.com/2006/11/latex-headheight-is-too-small.html
\setlength{\headheight}{15pt}

% paketi za pisanje algoritama
%\usepackage{algorithmic}
\usepackage[croatian,	linesnumbered, onelanguage]{algorithm2e}

% EUROPSKI NACIN RAZDVAJANJA ODLOMAKA
%\setlength{\parskip}{1ex plus 0.5ex minus 0.2ex}
%\setlength{\parindent}{0pt}

% indentacija na prvom paragrafu u poglavlju
\usepackage{indentfirst}

% macroi, naredbe i slicno
\usepackage{xparse}


% podesi prored na 1.5
\usepackage[onehalfspacing]{setspace}

% dodaje random tekst
\usepackage{lipsum}

% paketi za referenciranje
%\usepackage{varioref}
%\usepackage{cleveref}

% BIBLIOGRAFIJA

% koristiti biber s dodatkom za hrvatski jezik
% kod kompajliranja dokumenta!!!

% paket za upravljanje bibliografijom
%\usepackage[style=authoryear,citestyle=authoryear,doi=false,isbn=false,dashed=false,
%  maxcitenames=2,maxbibnames=100]{biblatex}
\usepackage[backend=bibtex,
  doi=true,
	isbn=true,
	url=false,
	maxbibnames=100,
	defernumbers=true,
	sorting=nyt,
	style=ieee,
	dashed=false,
	citestyle=numeric-comp]{biblatex}
\AtEveryBibitem{\clearfield{note}}

% pogledati 
% https://tex.stackexchange.com/questions/111363/exclude-fullcite-citation-from-bibliography
% za objasnjenje sljedece tri linije
% korisno ako zelimo printati s \fullcite bibligrafiju u zivotopisu
% a taj se clanak ne pojavljuje citiran u ostatku disertacije
\DeclareBibliographyCategory{fullcited}
\newcommand{\mybibexclude}[1]{\addtocategory{fullcited}{#1}}

% Izgled stranica (header, footer, ...)

\fancypagestyle{plain}{
	\fancyhf{} 
	\renewcommand{\headrulewidth}{0pt}
	\renewcommand{\footrulewidth}{0pt}
	\fancyfoot[R]{\thepage} 

	% sljedeća linija dodaje u footer informaciju o trenutnoj verziji dokumenta
	% potrebno ju je zakomentirati kod finalnog ispisa
	\fancyfoot[L]{\nouppercase{\footnotesize \today : version \documentVersion}} 
}

\pagestyle{fancy}{
	\fancyhf{} 
	%\renewcommand{\headrulewidth}{0pt}
	%\renewcommand{\footrulewidth}{0pt}
	\rhead{\nouppercase{\footnotesize \leftmark}}
	\fancyfoot[R]{\thepage} 

	% sljedeća linija dodaje u footer informaciju o trenutnoj verziji dokumenta
	% potrebno ju je zakomentirati kod finalnog ispisa
	\fancyfoot[L]{\nouppercase{\footnotesize \today : version \documentVersion}} 
}
\pagestyle{fancy}

% potrebno za dodavanje naslovnice u finalni dokument
\usepackage{pdfpages}

% postavi velicinu razmaka nakon znaka '.' (tocka) 
% uvijek na istu vrijednost. Inace ce razmak biti
% veci na kraju recenice nego kod skracenica npr.
%\frenchspacing

% korisno za provjeravanje dokumenta u finalnoj fazi
%\usepackage{refcheck}
%\usepackage{showkeys}

% korisno za mjenjanje jedne stranice u landscape mode
\usepackage{pdflscape}

%appendix da se dobro prikazuje u popisu i u naslovu kod samog appendixa umjesto Chapter
\usepackage[titletoc]{appendix}

% paket za stvaranje poveznica u elektronskom
% dokumentu. Koristan kod objavljivanja elektronskog
% oblika dokumenta
\usepackage[final]{hyperref}

\PassOptionsToPackage{unicode}{hyperref}
\PassOptionsToPackage{naturalnames}{hyperref}

\hypersetup{
	unicode=true,       % non-Latin characters in Acrobat’s bookmarks
	pdfauthor={\AuthorNameSurname},
	pdftitle={\ThesisTitleEng},
	pdfsubject={},
	pdfkeywords={\ThesisKeywords},
	colorlinks=true,    % false: boxed links; true: colored links
	linkcolor={black},  % color of internal links (change box color with linkbordercolor)
	citecolor={black},  % color of links to bibliography
	urlcolor={black}    % color of external links
}

% za review
\usepackage{soul} % precrtaj korisit \st{}
\newcommand{\remove}[1]{\textcolor{red}{}}
\newcommand{\add}[1]{\textcolor{blue}{#1}}