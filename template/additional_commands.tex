%%%%%%%%%%%%%%%%%%%%%%%%%%%%%%%%%%%%%%%%%%%%%%%%%%%%%%%%%%%%%%%%%%%%%%%%%%%%%%%%
% KORISNE KRATICE (teoremi, definicije, ...)
%%%%%%%%%%%%%%%%%%%%%%%%%%%%%%%%%%%%%%%%%%%%%%%%%%%%%%%%%%%%%%%%%%%%%%%%%%%%%%%%

\newtheoremstyle{stil}{}{}{}{}{\bfseries}{}{ }{}
\theoremstyle{stil}
\newtheorem{tm}{Teorem}[chapter]
\newtheorem{defn}[tm]{Definicija}
\newtheorem{pro}[tm]{Propozicija}
\newtheorem{prop}[tm]{Propozicija}
\newtheorem{lem}[tm]{Lema}
\newtheorem{kor}[tm]{Korolar}

%%%%%%%%%%%%%%%%%%%%%%%%%%%%%%%%%%%%%%%%%%%%%%%%%%%%%%%%
% KORISNE KRATICE (komande)
%%%%%%%%%%%%%%%%%%%%%%%%%%%%%%%%%%%%%%%%%%%%%%%%%%%%%%%%

% za dodavanje todo dijelova
\newcommand{\todo}[1]{\textcolor{red}{#1}}
%za citirane dijelove i dijelove pod navodnicima
\newcommand{\udquote}[1]{``\textit{#1}''}
\newcommand{\udquotehr}[1]{„#1”}

% koristimo kod prvog pojavljivanja pojma
\newcommand{\be}[1]{\textbf{\em{#1}}}

% konvergencija
\newcommand{\konv}[2]{#1 \rightarrow #2}

% konvergencija gotovo sigurno
\newcommand{\konvgs}[2]{#1 \xrightarrow{g.s.} #2}
\newcommand{\nkonvgs}[2]{#1 \not \xrightarrow{g.s} #2}
% konvergencija po vjerojatnosti
\newcommand{\konvpv}[2]{#1 \xrightarrow{P} #2}
\newcommand{\nkonvpv}[2]{#1 \not \xrightarrow{P} #2}
% konvergencija po distribuciji
\newcommand{\konvpd}[2]{#1 \xrightarrow{D} #2}
\newcommand{\nkonvpd}[2]{#1 \not \xrightarrow{D} #2}
% konvergencija u srednjem
\newcommand{\konvusr}[3]{#1 \xrightarrow{L^{#3}} #2}
\newcommand{\nkonvusr}[3]{#1 \not \xrightarrow{L^{#3}} #2}

% skalar
\newcommand{\ts}[1]{\lowercase{#1}}
% vektor
\newcommand{\tv}[1]{\mathbf{\lowercase{#1}}}
% matrica 
\newcommand{\tmatrix}[1]{\uppercase{#1}}
% tenzor
\newcommand{\ttenzor}[1]{\mathbf{\uppercase{#1}}}
% slucajna varijabla
\newcommand{\tsvar}[1]{\uppercase{#1}}
% slucajan vektor
\newcommand{\tsvek}[1]{\uppercase{\textbf{#1}}}

% niz znakova od interesa
\newcommand{\nz}[1]{{\sc #1}}
% specijalno za DNA kao hack da ih prvo spusti
% jer sam ih pisao velikim slovima
\newcommand{\nzdna}[1]{{\sc{\lowercase{#1}}}}
% specijalno za slicnost stringova
% jer se small caps ne prikazuje najbolje u math modu
\newcommand{\nzk}[1]{{\textit{#1}}}

% povlaka nad slovom u math modu
\newcommand{\ol}[1]{\widetilde{#1}}

% dupla povlaka nad slovom
% u math modu
\newcommand{\dol}[1]{\ol{\ol{#1}}}

% povlaka nad slovom u textmodu
\newcommand{\tol}[1]{\~{#1}}

%%%%%%%
% vjerojatnosni simboli
\newcommand{\vP}{\mathrm{P}}
\newcommand{\vE}[1]{\mathrm{E}\left(#1\right)}
\newcommand{\vVar}[1]{\mathrm{Var}\left(#1\right)}
\newcommand{\vCov}[1]{\mathrm{Cov}\left(#1\right)}


%%%%%%%
% matematicki skupovi
\newcommand{\setN}{\mathbb{N}}
\newcommand{\setZ}{\mathbb{Z}}
\newcommand{\setQ}{\mathbb{Q}}
\newcommand{\setR}{\mathbb{R}}
\newcommand{\setC}{\mathbb{C}}
\newcommand\scalemath[2]{\scalebox{#1}{\mbox{\ensuremath{\displaystyle #2}}}}


%%%%%%%
% neke funkcije/operatori
\DeclareMathOperator{\tg}{tg}
\DeclareMathOperator{\ctg}{ctg}
\DeclareMathOperator{\arctg}{arctg}
\DeclareMathOperator{\arcctg}{arcctg}
\DeclareMathOperator{\sh}{sh}
\DeclareMathOperator{\tgh}{th}
\DeclareMathOperator{\cth}{cth}
\DeclareMathOperator{\Ker}{Ker}
\DeclareMathOperator{\slika}{Im}
\DeclareMathOperator{\rang}{rang}
\DeclareMathOperator{\vek}{vec}
\DeclareMathOperator{\Span}{span}

%%%%%%%
% ostalo
\newcommand{\mbold}[1]{\mathbf{#1}}
\newcommand{\diag}{\mathop{\mathrm{diag}}}

